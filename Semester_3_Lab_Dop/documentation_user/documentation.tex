\documentclass[12pt]{article}

\usepackage[utf8]{inputenc}
\usepackage[T2A]{fontenc}
\usepackage[russian]{babel}

\usepackage{amsmath, amssymb}

\usepackage{xcolor}
\usepackage{listings}
\usepackage{caption}
\captionsetup[lstlisting]{labelfont=bf, labelsep=period}

% -------------------- Listings: ProbabilityScript --------------------
\lstdefinelanguage{ProbabilityScript}{
  morekeywords={repeat,collect,print,print_stat,if},
  sensitive=true,
  morecomment=[l]{//},
  morestring=[b]"
}

\lstset{
  language=ProbabilityScript,
  basicstyle=\ttfamily\small,
  columns=fullflexible,
  frame=single,
  framerule=0.4pt,
  rulecolor=\color{black!35},
  backgroundcolor=\color{black!3},
  numbers=left,
  numberstyle=\tiny\color{black!55},
  numbersep=8pt,
  xleftmargin=1.6em,
  tabsize=4,
  showstringspaces=false,
  breaklines=true,
  breakatwhitespace=true,
  captionpos=b
}

\begin{document}

% -------------------- Title page --------------------
\begin{titlepage}
\begin{center}
\large{\textbf{НИЯУ МИФИ}}\\
\large{\textbf{Кафедра 22 «Кибернетики»}}\\[5cm]

\textbf{\Large Лабораторная работа}\\[0.2cm]
\textbf{\large на тему: «Разработка компактного языка для вероятностного моделирования»}\\
\textbf{\large (язык сценариев ProbabilityScript)}\\[5cm]

\begin{flushright}
Выполнил: \underline{\hspace{5cm}} (ФИО студента)\\[0.3cm]
Руководитель: \underline{\hspace{5cm}} (ФИО преподавателя)\\
\end{flushright}

\vfill
Москва -- 2025
\end{center}
\end{titlepage}

% -------------------- Intro --------------------
\section*{Введение}
\addcontentsline{toc}{section}{Введение}
Цель проекта -- разработка компактного предметно-ориентированного языка \textbf{ProbabilityScript} для вероятностного моделирования.
Язык сценариев предназначен для упрощения моделирования случайных процессов и статистических вычислений.
В рамках проекта реализован интерпретатор, позволяющий задавать последовательность случайных экспериментов, собирать результаты и рассчитывать основные статистические показатели.

Представленная документация описывает синтаксис и возможности языка ProbabilityScript, архитектуру реализации интерпретатора, пример использования языка, а также содержит заключение с выводами и возможными направлениями развития.
В приложении приведён полный листинг демонстрационного сценария \texttt{demo\_basic.psc}, иллюстрирующего основные возможности языка.

% -------------------- Syntax --------------------
\section{Синтаксис языка ProbabilityScript}
Данный раздел описывает синтаксис и ключевые возможности языка сценариев ProbabilityScript.
Язык имеет минималистичный синтаксис и предназначен для генерации случайных чисел, повторения серии экспериментов, условий, а также вычислений и сбора статистики.
Основной (и единственный) тип данных в языке -- вещественное число с плавающей запятой (floating point).

\subsection{Типы данных}
В языке ProbabilityScript используется один базовый тип данных:
\begin{itemize}
    \item \textbf{Вещественные числа:} все переменные и выражения являются числовыми (тип с плавающей запятой, обычно двойной точности). Целочисленные значения не выделяются отдельно, а представляют собой частный случай вещественного.
\end{itemize}

Переменные в сценарии не объявляются заранее -- они неявно создаются при первом присваивании.
Имена переменных могут состоять из букв латинского алфавита, цифр и символа подчёркивания, начинаются с буквы или подчёркивания.
Примеры допустимых имён: \texttt{x}, \texttt{value1}, \texttt{sum\_2}.

\subsection{Выражения и операторы}
Выражения используются в присваиваниях, \texttt{print}, \texttt{collect}, условиях \texttt{if} и аргументах функций.
Поддерживаются:
\begin{itemize}
    \item \textbf{Числа:} \texttt{1}, \texttt{3.14}.
    \item \textbf{Переменные:} \texttt{x}, \texttt{sum}.
    \item \textbf{Вызовы функций:} \texttt{uniform(0, 1)}, \texttt{mean()}.
    \item \textbf{Унарный минус:} \texttt{-x}, \texttt{-(1 + 2)}.
    \item \textbf{Скобки:} \texttt{( ... )} для явного задания порядка вычислений.
\end{itemize}

\medskip
\noindent\textbf{Арифметические операторы:}
\[
+, \; -, \; *, \; /
\]
Они вычисляют результат как обычные арифметические операции над вещественными числами.

\medskip
\noindent\textbf{Операторы сравнения:}
\[
>, \; <, \; ==, \; !=
\]
Результат сравнения является числом: \texttt{1.0}, если условие истинно, и \texttt{0.0}, если ложно.
Это важно для конструкции \texttt{if}, которая проверяет «ненулевость» результата.

\medskip
\noindent\textbf{Приоритеты операций (от более высокого к более низкому):}
\begin{enumerate}
    \item унарный минус \texttt{-}
    \item умножение/деление \texttt{*}, \texttt{/}
    \item сложение/вычитание \texttt{+}, \texttt{-}
    \item операции сравнения \texttt{>}, \texttt{<}, \texttt{==}, \texttt{!=}
\end{enumerate}

\subsection{Конструкции языка}
Язык сценариев предоставляет небольшой набор управляющих конструкций и инструкций:
\begin{itemize}
    \item \textbf{Присваивание.} Оператор \texttt{=} используется для присваивания значения переменной: \texttt{x = 5}, \texttt{y = x + 1.5}.
    \item \textbf{Цикл \texttt{repeat}.} Синтаксис: \texttt{repeat $E$ \{ ... \}}, где $E$ -- числовое выражение (ожидается неотрицательное значение; фактически используется целая часть). Блок в фигурных скобках выполняется указанное число раз. Вложенные блоки и вложенные \texttt{repeat} допускаются.
    \item \textbf{Условие \texttt{if}.} Синтаксис: \texttt{if $E$ \{ ... \}}, где $E$ -- числовое выражение. Если $E \neq 0$, блок выполняется, иначе пропускается. Обычно в условии используются сравнения, например: \texttt{if x > 0 \{ ... \}}.
    \item \textbf{Инструкция \texttt{collect}.} \texttt{collect $E$} вычисляет выражение $E$ и добавляет значение в выборку (по умолчанию в единственную текущую выборку).
    \item \textbf{Вывод \texttt{print}.} \texttt{print $E$} выводит значение выражения $E$ в стандартный вывод. Команда печатает число и перевод строки.
    \item \textbf{Вывод статистики \texttt{print\_stat}.} \texttt{print\_stat("name")} печатает \textbf{одну} статистику по текущей выборке. Поддерживаются: \texttt{"mean"}, \texttt{"variance"} (или \texttt{"var"}), \texttt{"stddev"} (или \texttt{"std"}), \texttt{"median"}, \texttt{"count"}.
\end{itemize}

Блоки кода ограничиваются фигурными скобками \texttt{\{\}}.
Каждая инструкция записывается с новой строки (точка с запятой не требуется).
Поддерживаются однострочные комментарии \texttt{// ...}.

\subsection{Встроенные функции}
Для организации вероятностных экспериментов и статистической обработки в языке ProbabilityScript доступны следующие встроенные функции:

\begin{itemize}
    \item \textbf{\texttt{uniform()}}: равномерное распределение по умолчанию $U(0, 1)$.
    \item \textbf{\texttt{uniform($b$)}}: равномерное распределение $U(0, b)$.
    \item \textbf{\texttt{uniform($a$, $b$)}}: равномерное распределение $U(a, b)$, где должно выполняться $a < b$.

    \item \textbf{\texttt{normal()}}: нормальное распределение по умолчанию $N(0, 1)$.
    \item \textbf{\texttt{normal($\mu$)}}: нормальное распределение $N(\mu, 1)$.
    \item \textbf{\texttt{normal($\mu$, $\sigma$)}}: нормальное распределение $N(\mu, \sigma)$, где $\sigma > 0$.

    \item \textbf{\texttt{mean()}} – среднее по текущей выборке.
    \item \textbf{\texttt{variance()}} (или \texttt{var()}) – дисперсия по текущей выборке.
    \item \textbf{\texttt{stddev()}} (или \texttt{std()}) – стандартное отклонение по текущей выборке.
    \item \textbf{\texttt{median()}} – медиана по текущей выборке.
    \item \textbf{\texttt{count()}} – размер текущей выборки.
\end{itemize}

Функции \texttt{mean}, \texttt{variance}, \texttt{stddev}, \texttt{median}, \texttt{count} не принимают аргументов.
Если выборка пуста, вызов статистики считается ошибкой исполнения (в текущей реализации).

\subsection{Примеры сценариев}
Ниже приведены примеры использования ProbabilityScript.

\medskip
\noindent\textbf{Пример 1. Оценка характеристик равномерного распределения.}
\begin{lstlisting}[caption={Оценка характеристик равномерного распределения}, label={lst:uniform_stats}]
N = 100
repeat N {
    x = uniform(0, 1)
    collect x
}
print_stat("count")
print_stat("mean")
print_stat("variance")
print_stat("stddev")
print_stat("median")
\end{lstlisting}

\medskip
\noindent\textbf{Пример 2. Условие \texttt{if} и операторы сравнения.}
\begin{lstlisting}[caption={Условие if и операторы сравнения}, label={lst:if_cmp}]
x = 1 + 2
if x > 2 {
    print 42
}
if x == 0 {
    print 99
}
\end{lstlisting}

В этом примере первое условие истинно (печатается \texttt{42}), второе -- ложно (блок не выполняется).

\medskip
\noindent\textbf{Пример 3. Оценка характеристик нормального распределения.}
\begin{lstlisting}[caption={Оценка характеристик нормального распределения}, label={lst:normal_stats}]
N = 50
repeat N {
    y = normal(0, 1)
    collect y
}
print mean()
print stddev()
\end{lstlisting}

% -------------------- Architecture --------------------
\section{Архитектура реализации}
Реализация интерпретатора ProbabilityScript основана на классической структуре интерпретируемого языка:
входной сценарий последовательно проходит этапы лексического анализа, синтаксического анализа (разбора),
построения внутреннего представления и непосредственного исполнения.

Основные компоненты системы: лексер, парсер, дерево разбора (AST), интерпретатор, среда выполнения и модуль статистики.

\subsection{Модули интерпретатора}
\begin{itemize}
    \item \textbf{Лексер (Lexer).} Преобразует исходный текст в поток токенов: ключевые слова (\texttt{repeat}, \texttt{collect}, \texttt{print}, \texttt{if}),
    идентификаторы, числа, строковые литералы (для \texttt{print\_stat("...")}),
    знаки операций и скобки. Поддерживаются однострочные комментарии \texttt{// ...}.
    \item \textbf{Парсер (Parser).} Строит AST по потоку токенов.
    Реализованы правила для операторов (\texttt{repeat}, \texttt{if}, \texttt{collect}, \texttt{print}, \texttt{print\_stat}, присваивание),
    а также выражений с поддержкой арифметики и сравнений.
    \item \textbf{AST.} Представляет программу как дерево узлов выражений и операторов (в т.ч. \texttt{IfStmt} и \texttt{BinaryExpr} для сравнений).
    \item \textbf{Интерпретатор.} Выполняет AST. Сравнения возвращают \texttt{1.0/0.0}, а \texttt{if} проверяет условие как «значение не равно нулю».
    \item \textbf{Среда выполнения (Environment).} Хранит переменные и собранную выборку значений (для \texttt{collect}).
    \item \textbf{Модуль статистики.} Содержит расчёт статистик по выборке: \texttt{mean}, \texttt{variance}, \texttt{stddev}, \texttt{median}, \texttt{count}.
\end{itemize}

\subsection{Путь от сценария до исполнения}
После запуска интерпретатора с указанием сценария ProbabilityScript выполнение происходит по следующим этапам:
\begin{enumerate}
    \item Чтение файла сценария (\texttt{.psc}) в строку.
    \item Лексический анализ: текст преобразуется в токены.
    \item Синтаксический анализ и построение AST.
    \item Исполнение AST интерпретатором: вычисление выражений, выполнение \texttt{repeat} и \texttt{if},
    сбор данных через \texttt{collect}, вывод через \texttt{print} и \texttt{print\_stat}.
    \item Завершение работы (или сообщение об ошибке при некорректном сценарии/ошибке исполнения).
\end{enumerate}

% -------------------- Usage --------------------
\section{Пример использования}
\begin{enumerate}
    \item \textbf{Подготовка сценария.} Сценарий хранится в файле с расширением \texttt{.psc}.
    \item \textbf{Запуск интерпретатора.} Пример:
\begin{verbatim}
$ psc demo_basic.psc
\end{verbatim}
    \item \textbf{Фиксация seed (воспроизводимость).} Пример:
\begin{verbatim}
$ psc --seed=42 demo_basic.psc
\end{verbatim}
\end{enumerate}

% -------------------- Conclusion --------------------
\section*{Заключение}
\addcontentsline{toc}{section}{Заключение}
В рамках проекта разработан и реализован простой интерпретируемый язык сценариев \textbf{ProbabilityScript} для вероятностного моделирования.
Язык поддерживает генерацию случайных чисел из базовых распределений (равномерного и нормального),
повторение серий экспериментов с накоплением результатов, вычисление основных статистических характеристик выборки,
условный оператор \texttt{if} на базе числовой семантики истинности,
а также выражения с арифметическими операциями и операторами сравнения.

Дальнейшее развитие может включать добавление \texttt{else}, логических операций (\texttt{\&\&}, \texttt{||}),
новых распределений, поддержку нескольких именованных выборок, расширение типов данных и более удобные средства вывода/экспорта результатов.

% -------------------- Appendix --------------------
\appendix
\section*{Приложение: Полный листинг сценария \texttt{demo\_basic.psc}}
\addcontentsline{toc}{section}{Приложение: Листинг сценария demo\_basic.psc}

\begin{lstlisting}[caption={Сценарий demo\_basic.psc}, label={lst:demo_basic}]
// demo_basic.psc

N = 100
repeat N {
    x = uniform(0, 1)
    collect x
}
print_stat("count")
print_stat("mean")
print_stat("variance")
print_stat("stddev")
print_stat("median")
\end{lstlisting}

\end{document}
